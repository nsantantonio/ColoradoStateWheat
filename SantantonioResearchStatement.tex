\documentclass[11pt]{article}
\usepackage{amsmath}
\usepackage{amssymb}
\usepackage[backend=bibtex, style=authoryear]{biblatex}

\textwidth=470pt
\oddsidemargin=0pt
\topmargin=0pt
\headheight=0pt
\textheight=650pt
\headsep=0pt

\title{Statement of Research}


\author{Nicholas Santantonio}
\date{\today}


\newcommand{\gxe}{G$\times$E}
\newcommand{\gxg}{G$\times$G}


\begin{document}

\section*{\centering Statement of Research}
\begin{center} Nicholas Santantonio \end{center}


\noindent Winter wheat comprises a substantial portion of Colorado's' agriculture output. I aim to implement the latest breeding technology in the wheat breeding program to acheive population improvement alongside varietal development to maximize product output for Colorado growers. I intend to develop the foundational capacity to implement $21^\text{st}$ century breeding technology in the Wheat Breeding Program at Colorado State University, providing ample opportunities for extramural funding. This will include the collection, storage and accessibility of genome-wide information, pedigrees, high-throughput phenotypes collected through proximal sensing, and traditional phenotypes from field trials. Ground-breaking breeding technologies will be implemented to demonstrate the potential of these new resources and pique public interest in crop improvement at CSU.  

\subsection*{Short-term gains}

% I intend to use wheat and barley as model organisms to ask important questions about how to transition a traditional breeding program into a $21^\text{st}$ century breeding program

The most easily exploited terms in the breeders equation are cycle time and selection intensity. The number of lines evaluated can be drastically increased by reducing or eliminating replication in early-stage trials, thus increasing selection intensity. Genotyping some lines within each family will increase reliability and allow for estimation of genetic correlations of locations and prediction of genetic merit within locations. All materials that are advanced to the second year yield trials will be genotyped. Aerial phenotypes obtained through proximal sensing will be used in multi-trait models to substantially increase trial sizes and reduce the number of plots that are harvested. High-throughput phenotypes will also be used to account for spatial variation within the field for better estimation of genetic values of unreplicated lines, as well as monitor diseases.% \emph{Fusarium} head blight (scab). 

Initially, the breeding cycle will be shortened by recycling materials after the second year of yield trials, using genotypic information and mathematical optimization to drive crossing decisions. A Smith-Hazel economic selection index (or indices) including yield and test weight will be constructed by talking with regional grain elevator operators, which will be used as the genetic merit criterion to transition to a optimal contribution genomic selection program. Other properties, including milling and baking quality in wheat, forage and malting quality in barley, straw content and quality, as well as other disease resistances will all be monitored and included in the decision making process. 

\subsection*{Long-term sustainability}

The infrastructure to support regular genotyping and proximal sensing must be developed before the program is overwhelmed by incoming data. While I intend to host a local breeding program database, I would like to work with the T3 database to host proximal sensing images along with phenotypic trial data, pedigrees and genotypes. Quality control of genetic markers, pedigrees and phenotypes must be implemented and standard operating procedures must be developed. Genotyping will allow for construction of a multi-trait, multi-environment genomic prediction model that will be updated yearly. Field experimental design will then be optimized to leverage all phenotypic information available, using genotypic information to link otherwise unrelated trials and traits.

% All phenotypic information will be be connected through pedigrees and genome-wide markers. Predictive models will be updated with new individuals and phenotypes on a yearly basis.

Once genotyping and rapid recycling has been hammered out and sufficient genotypic and phenotypic information is generated ($\leq 5$ years), a recurrent population will be constructed using current breeding materials. This population will be subjected to rapid cycling twice a year to increase the genetic merit of the population, generate unique meiotic recombinations and minimize inbreeding using optimal contribution. Materials will be pulled out of the recurrent population on a yearly basis for two generations of high intensity genomic selection and two generations of inbreeding in the greenhouse. Family sizes will vary based on predicted family means and variances. Materials will also be pulled directly out of the recurrent population and phenotyped for generating information highly connected to the recurrent population to increase accuracy of selection. 


Specialty wheats for specific flour types.  


\subsection*{Breeding for synergistic organism interactions}

A legume rotational crop with wheat could drastically improve soil health, nitrogen and long-term sustainability. The wheat-soybean double cropping system has allowed many farmers in the Midwest to profit from growing winter wheat by production of a late season soybean crop, but breeding efforts have largely focused on improving each crop independently. I intend to investigate the implementation of a legume double cropping system for Colorado using field peas. By breeding both organisms for synergistic combinations well suited for the tight double cropping system, I hope to produce variety pairs that mature early, yield well and sustain adequate soil moisture for the more valuable wheat crop. 

% Important wheat traits would include fall vigor, early maturity and uniform nitrogen use, while important soybean traits include establishment in wheat stubble and early late-season maturity, along with all the other agronomic traits that make these crops viable options for growers. Similarly, an economic index comprising of both the wheat and soybean output could be constructed to maximize total season profitability. 

While the number of potential wheat-pea genotype pairs is very large and generally intractable, clever testing of select pairs will allow for genomic prediction of unobserved combinations. This is similar to the G$\times$E problem, where one crop (e.g. pea) can be thought of as an environment to which the genotype (e.g. wheat) is subjected to. In this case, the genetic correlation of genotypes \emph{and} environments would be estimated with genome-wide markers. All pairwise combinations would then be predicted using multivariate linear mixed models. Promising predicted pairs would then be evaluated the following season, and crosses made to maximize the beneficial interaction effect. %While each round of genomic interaction selection would take at least two years to complete depending on the number of trial years desired, the process could improve both crops toward one another. 

% While the genetic covariance of environments is generally unknown and must be estimated \emph{post hoc}, in this case, \emph{a priori} estimates of genetic covariance between genotypes (say for wheat is $\mathbf{A}$) and ``environments'' (say for soybeans is $\mathbf{B}$) can be obtained using genome-wide markers. When both covariances are known, the covariance of all combinations is simply $\mathbf{A} \otimes \mathbf{B}$, 


These synergistic genomic selection models could also be adapted to other goals, allowing for much wider collaborations between animal breeders, microbiologists and soil scientists. Examples include root/soil microbiome using wheat as a model, forage/animal and forage/rumen-microbiome using alfalfa as a model, and even barley/yeast interactions using malting barley as a model. Products specifically improved for one another could be marketed as a package (e.g. maltsters could offer specific yeasts with certain barley malts), or a service (e.g. genotyping a farmers field to determine optimal varieties for their soil microbiome). Synergistic improvement of multiple organisms has only just recently become feasible due to cost-effective genotyping platforms, and has the potential to change the way we breed within our food systems. 



\subsection*{Research Philosophy}

In the era of big data, a shift away from small designed experiments to large observational studies at the breeding program scale is inevitable. When genotyped and made publicly available with FAIR data principals, the vast amount of data generated in a breeding program becomes a treasure trove for asking questions and informing breeding decisions. Clever experimental design will offset genotyping costs by trading replication at an individual level for replication at the genetic level.

I believe in a collaborative model, where breeding programs do not operate in isolation. They share germplasm, resources, expertise, and most importantly, ideas. I intend to contribute to the collaborative effort at CSU and across the globe to build the foundational capabilities needed to deploy the latest technology for variety development. As climate change progresses, heat, drought, intense storms and hard frosts will be the new norm, and we must work together to help defend our food security through accelerated genetic improvement. 


% In the era of big data, a shift away from small designed experiments to large observational studies at the breeding program scale is inevitable. When genotyped, stored, and made publicly available with FAIR data principals, the vast amount of data generated in a breeding program becomes a treasure trove for asking questions and informing breeding decisions. Genotyping at this scale is feasible given the drastic reduction in cost, and can be further offset by clever experimental design that trades replication at an individual level for replication at the genetic level.
% % A traditional breeding program generates a vast amount of phenotypic data that is used to make yearly breeding decisions, and subsequently discarded
% % This does not mean that we should cease the design and execution of experiments to address specific hypotheses, but we cannot ignore the valuable resource of observational data being collected, typically at great expense. 

% I believe in a collaborative model, where breeding programs do not operate in isolation. They share germplasm, resources, expertise, and most importantly, ideas. I intend to contribute to the collaborative effort at CSU and across the globe to build the foundational capabilities needed to deploy the latest technology for variety development. As climate change progresses, heat, drought, intense storms and hard frosts will be the new norm, and we must work together to help defend our food security through accelerated genetic improvement. 


\end{document}
