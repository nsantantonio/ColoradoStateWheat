
\pagenumbering{gobble}
\documentclass[11pt, letterpaper]{moderncv}
\usepackage{times, amsmath}

\moderncvstyle{classic}
\moderncvcolor{green}
\usepackage[utf8]{inputenc}

\usepackage[scale=0.75, margin = 0.75in]{geometry}
\geometry{top  = 20mm}

\textwidth=470pt
\oddsidemargin=0pt
\topmargin=0pt
\headheight=0pt
\textheight=650pt
\headsep=0pt

\pdfpagewidth=\paperwidth
\pdfpageheight=\paperheight


\name{Nicholas}{Santantonio}
\address{240 Emerson Hall, Ithaca, NY 14853}{(505)~412-2738}
\extrainfo{ns722@cornell.edu}

\begin{document}
\recipient{Department of Soil \& Crop Sciences}{College of Agricultural Sciences\\Colorado State University\\ Fort Collins, CO 80523}
\date{\today}
\opening{Dear Hiring Committee,}
\closing{Respectfully yours,}
\makelettertitle

I am eager to apply to the Assistant Professor of Wheat Breeding and Genetics position in the Soil and Crop Sciences Department at Colorado State University. I am currently a postdoctoral associate at Cornell University, working with Dr. Kelly Robbins on quantitative genetics solutions for plant breeding. I have combined a strong applied background in small grains and forage breeding with theoretical quantitative genetics, allowing me to integrate the newest computational technologies into a working breeding program.

I have demonstrated commitment and determination in research, teaching and leadership. I have obtained extra-mural funding to pursue the integration of digital ag and population-based genomic selection strategies for alfalfa improvement. To further my teaching experience and hone my philosophy, I co-instructed an advanced graduate-level quantitative genetics course during my postdoc. Serving as a co-leader for the Diversity Preview Weekend at Cornell, I have shown a dedication to learning about, and working toward a diverse, equitable and inclusive academic environment. 

By integrating genome-wide information and proximal sensing, I aim to accelerate product development for Colorado farmers by increasing selection intensity and pushing generation turnover times toward the biological limits of wheat. I intend to use wheat as model to demonstrate how to effectively transition to a data-driven, $21^\text{st}$ century breeding program. Working closely with the new quantitative genomicist, I hope to address both theoretical and logistical implementation problems, while seeking collaborators across the nation and around the globe. This transition will provide a valuable resource for public outreach, where farmers and consumers can learn how we are adapting the latest technologies to help fortify our food systems.

Moving forward, I want to establish ties across CSU and with Colorado farmers to build a cropping systems integration initiative, where all parts of the agronomic ecosystem are considered when making breeding decisions. I aim to help prepare future students for data-driven plant breeding by teaching courses with a quantitative, hands-on approach. Most importantly, I intend to pursue several initiatives to develop community based outreach programs, create a diverse experience requirement for graduate students, and shed light on the hidden curriculum in academia. I am dedicated to cultivating a safe, inclusive and equitable environment where students and staff of all backgrounds can thrive. 

I thank the hiring committee for considering my application for the Assistant Professor of Wheat Breeding and Genetics. CSU has a rich legacy in plant breeding to which I hope to contribute. 

\vspace{3mm}

Sincerely,\\
\vspace{2cm}
Nicholas Santantonio

\end{document}

