\documentclass[11pt]{article}
\usepackage{amsmath}
\usepackage{amssymb}
\usepackage[backend=bibtex, style=authoryear]{biblatex}


\usepackage[mmddyyyy,hhmmss]{datetime}
\usepackage{xcolor}

\definecolor{hyperblue}{rgb}{0,0,0.7}

\usepackage{hyperref}[hidelinks]

\hypersetup{
	colorlinks = true, 
	urlcolor = hyperblue, 
	linkcolor = hyperblue, 
    anchorcolor = hyperblue,
    citecolor = hyperblue,
    filecolor = hyperblue,
     }

\definecolor{nicholasCol}{RGB}{203,97,63}
\definecolor{amyCol}{RGB}{255,20,147}
\newcommand{\nicholas}[1]{{\color{nicholasCol} [\textbf{NS:} #1 (\today\ \currenttime)]}}
\newcommand{\amy}[1]{{\color{amyCol} [\textbf{Amy:} #1]}}


\textwidth=470pt
\oddsidemargin=0pt
\topmargin=0pt
\headheight=0pt
\textheight=650pt
\headsep=0pt

\pdfpagewidth=\paperwidth
\pdfpageheight=\paperheight

\setcounter{page}{5}

\title{Contribution to Diversity Equity and Inclusion}

\author{Nicholas Santantonio}
\date{\today}

\begin{document}


\section*{\centering Statement of Diversity, Equity and Inclusion}
\begin{center} Nicholas Santantonio \end{center}

\noindent \fbox{
\begin{minipage}{0.95 \linewidth}
	\emph{\textbf{Note to CSU:} I have spent much time considering these issues. You can find a detailed version of my thoughts and the initiatives I intend to pursue here:} \href{https://github.com/nsantantonio/ColoradoStateQG/blob/master/SantantonioDEIContributionLong.pdf}{\small{github.com/nsantantonio/ColoradoStateQG/blob/master/SantantonioDEIContributionLong.pdf}} 
\end{minipage}
}

\medskip

\noindent Everyone deserves an equal chance to show their potential. Often, it takes the right time, place, person or simply an opportunity to inspire someone. As a high school dropout, it took several tries for me. I struggled through the first two years of my undergrad at New Mexico State (NMSU), failing to meet a 2.0 GPA by the end of my fourth semester. With the help of several professors at NMSU and continued support from my family, I found a passion for genetics during my third year and everything changed.

How can I provide adequate support and opportunity to help create that spark in those far less privileged than I? As an educator, I can work to create opportunities for students from underrepresented groups to pursue further education and expose students to different cultural perspectives in and outside of the classroom. I can take active measures to include students as part of a healthy lab culture, where expectations are clear and everyone is given equal opportunity to succeed in their own pursuits. Importantly, I can continue my own education into diversity, equity and inclusion (DEI) so that I can adapt my own efforts to better serve the community. 

I have worked toward learning about, and engaging in DEI concepts and initiatives to further my understanding and to help me become a better ally. To actively engage in increasing the diversity of graduate students in the plant sciences, I joined the Diversity Preview Weekend at Cornell as a co-leader, currently serving as fundraising chair. I have attended several workshops on the hidden curriculum to learn how I can identify and help shed light on the unspoken expectations in academia that may be foreign to those unfamiliar with the academic environment in the US.

It is important that we recognize the need for education in DEI concepts and ideas. To enact societal change, individuals need to be willing to be taught and engage in the conversation. During this learning process many will make mistakes, including me. Mistakes are okay! As long as we learn from them and strive to better ourselves and our understanding of and compassion for others. It is paramount that we cultivate an environment where people can learn, practice and ask questions about DEI concepts without fear of retribution if they misstep. 

As faculty, we must educate ourselves on how to provide an inclusive environment and work to make resources equally available and expectations clear. We must reevaluate our own practices and implicit biases to ensure we have not unintentionally created an inequitable or exclusive situation. The curriculum needs to be built to reflect of the breadth of diversity, such that students recognize a bit of themselves, their culture or contributions that were made by people they can relate to. To address these needs, I want to work to develop a series of DEI related requirements for faculty, staff and graduate students, and engage with the community both on and off campus. Details of initiatives I aim to pursue can be found \href{https://github.com/nsantantonio/ColoradoStateQG/blob/master/SantantonioDEIContributionLong.pdf}{\textbf{here}}. 

The most effective teams with the best ideas come from diverse backgrounds and experiences. Women, minorities, LGBTQIA and members of other underrepresented groups continue to face less opportunity and numerous obstacles, especially in STEM. Unfortunately, this has lead to a lack of perceived academic ``merit'' by some, when these differences are purely environmental in nature and driven by legacy and current social biases. I am committed to helping reverse these trends by building a diverse team and an environment in which everyone is valued and given the resources and opportunities they need to succeed. 

\end{document}
